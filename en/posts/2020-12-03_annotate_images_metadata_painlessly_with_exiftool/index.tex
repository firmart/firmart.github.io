% Created 2020-12-19 sam. 23:31
% Intended LaTeX compiler: pdflatex
\documentclass[11pt]{article}
\usepackage[utf8]{inputenc}
\usepackage[T1]{fontenc}
\usepackage{graphicx}
\usepackage{grffile}
\usepackage{longtable}
\usepackage{wrapfig}
\usepackage{rotating}
\usepackage[normalem]{ulem}
\usepackage{amsmath}
\usepackage{textcomp}
\usepackage{amssymb}
\usepackage{capt-of}
\usepackage{hyperref}
\usepackage{minted}
\usepackage{parskip}
\usepackage[margin=2cm]{geometry}
\author{Firmin Martin}
\date{2020-12-03}
\title{Rating images painlessly with exiftool feat. ranger \& sxiv}
\hypersetup{
 pdfauthor={Firmin Martin},
 pdftitle={Rating images painlessly with exiftool feat. ranger \& sxiv},
 pdfkeywords={},
 pdfsubject={},
 pdfcreator={Emacs 28.0.50 (Org mode 9.4.2)}, 
 pdflang={English}}
\begin{document}

\maketitle
\tableofcontents


\section{Background}
\label{sec:orga8b07d5}
I was looking for a way to classifying images by rating them on the fly.
My first attempt was using \texttt{darktable} as suggested in \href{https://discuss.pixls.us/t/quick-rating-and-auto-advance-in-linux-a-la-photo-mechanic/3446}{a thread}. Indeed, the
auto-advance rating mechanism was quite handy. But it is still
too heavy for this sole purpose. In \texttt{darktable}, user have to import images before
editing metadata. When tens of thousands images are involved, the process of
importing images can be quite time-consuming\footnote{Dozen hours for 950.000 images.} as it creates for each image
an XMP file to store metadata. And this also applies to rating, even though I 
configure it to improve performance (without OpenCL) the latency is counted by seconds.
Moreover, XMP are also created for symlink of image. This was not plausible
in my use-case\footnote{Statistical classification of images. For each class, it creates
a directory in which each symbolic link is associated to the actual image.} as it enforces me to keep multiple metadata files for the same image.

Lots of critics, but clearly \texttt{darktable} was not the right tool. It suits
better on raw photo post-production as intended. I will present in this post two
solutions to remediate the issues aforementioned.

\section{Goals}
\label{sec:orga59b243}
After experiencing \texttt{darktable}, I know better what I am seeking:
\begin{enumerate}
\item \textbf{Edit metadata in the image file itself}. This has two advantages:
\begin{enumerate}
\item Keep metadata even if the filename is changed.
\item Get rid of XMP files.
\end{enumerate}
\item \textbf{Preview and select image without latency}. Namely, preview images and rate them on the fly.
\item \textbf{Metadata editing should follow symlink}. To centralize metadata in the same place.
\item \textbf{Batch rating}. Rating a whole directory or multiple selected images at once.
\end{enumerate}

\section{Exiftool}
\label{sec:org587c206}

\href{https://exiftool.org/}{ExifTool} is a free and open-source software program for reading, writing, and
manipulating image, audio, video, and PDF metadata. 
\subsection{Rating images with exiftool}
\label{sec:orge3f5f41}
Rating image with exiftool is very simple.
\begin{minted}[]{shell}
exiftool -rating=5 -overwrite_original_in_place <files>
\end{minted}
The option \texttt{-overwrite\_original\_in\_place} overwrite directly the file(s) instead
of moving the original one to \texttt{filename.ext\_original}. Use it wisely at your own risk.

To read back the rating:
\begin{minted}[]{shell}
exiftool -rating <files>
\end{minted}
\ldots{} or format yourself the output: 
\begin{minted}[]{shell}
exiftool -p '$Rating $Filepath' -f <files>
\end{minted}

And, of course the symbolic links are followed\footnote{Beware, without \texttt{-overwrite\_original\_in\_place}, \href{https://exiftool.org/forum/index.php?topic=6308.0}{symlink will be removed}!}!

But with exiftool alone, one cannot watch and rate image at the same time. 
This can be done by combine up exiftool with a file manager having preview ability
or an image viewer. Next, I will show how to integrate exiftool capability in
the file manager \texttt{ranger} and the image viewer \texttt{sxiv}.

\subsection{File types supported by exiftool}
\label{sec:org78aac9f}

In fact, the version \texttt{11.88} of exiftool already supports a large set of file types.
Thus, what has been and will be said is not limited to images and rating.

\begin{table}[htbp]
\centering
\begin{tabular}{|l|l|l|l|l|}
\hline
\texttt{3FR}   (r) & \texttt{DR4}   (r/w/c) & \texttt{ITC}   (r) & \texttt{ODP}   (r) & \texttt{RIFF}  (r)\\
\texttt{3G2}   (r/w) & \texttt{DSS}   (r) & \texttt{J2C}   (r) & \texttt{ODS}   (r) & \texttt{RSRC}  (r)\\
\texttt{3GP}   (r/w) & \texttt{DV}    (r) & \texttt{JNG}   (r/w) & \texttt{ODT}   (r) & \texttt{RTF}   (r)\\
\texttt{A}     (r) & \texttt{DVB}   (r/w) & \texttt{JP2}   (r/w) & \texttt{OFR}   (r) & \texttt{RW2}   (r/w)\\
\texttt{AA}    (r) & \texttt{DVR}-\texttt{MS} (r) & \texttt{JPEG}  (r/w) & \texttt{OGG}   (r) & \texttt{RWL}   (r/w)\\
\texttt{AAE}   (r) & \texttt{DYLIB} (r) & \texttt{JSON}  (r) & \texttt{OGV}   (r) & \texttt{RWZ}   (r)\\
\texttt{AAX}   (r/w) & \texttt{EIP}   (r) & \texttt{K25}   (r) & \texttt{OPUS}  (r) & \texttt{RM}    (r)\\
\texttt{ACR}   (r) & \texttt{EPS}   (r/w) & \texttt{KDC}   (r) & \texttt{ORF}   (r/w) & \texttt{SEQ}   (r)\\
\texttt{AFM}   (r) & \texttt{EPUB}  (r) & \texttt{KEY}   (r) & \texttt{OTF}   (r) & \texttt{SKETCH} (r)\\
\texttt{AI}    (r/w) & \texttt{ERF}   (r/w) & \texttt{LA}    (r) & \texttt{PAC}   (r) & \texttt{SO}    (r)\\
\texttt{AIFF}  (r) & \texttt{EXE}   (r) & \texttt{LFP}   (r) & \texttt{PAGES} (r) & \texttt{SR2}   (r/w)\\
\texttt{APE}   (r) & \texttt{EXIF}  (r/w/c) & \texttt{LNK}   (r) & \texttt{PBM}   (r/w) & \texttt{SRF}   (r)\\
\texttt{ARQ}   (r/w) & \texttt{EXR}   (r) & \texttt{LRV}   (r/w) & \texttt{PCD}   (r) & \texttt{SRW}   (r/w)\\
\texttt{ARW}   (r/w) & \texttt{EXV}   (r/w/c) & \texttt{M2TS}  (r) & \texttt{PCX}   (r) & \texttt{SVG}   (r)\\
\texttt{ASF}   (r) & \texttt{F4A}, \texttt{F4V} (r/w) & \texttt{M4A}, \texttt{M4V} (r/w) & \texttt{PDB}   (r) & \texttt{SWF}   (r)\\
\texttt{AVI}   (r) & \texttt{FFF}   (r/w) & \texttt{MAX}   (r) & \texttt{PDF}   (r/w) & \texttt{THM}   (r/w)\\
\texttt{AVIF}  (r/w) & \texttt{FITS}  (r) & \texttt{MEF}   (r/w) & \texttt{PEF}   (r/w) & \texttt{TIFF}  (r/w)\\
\texttt{AZW}   (r) & \texttt{FLA}   (r) & \texttt{MIE}   (r/w/c) & \texttt{PFA}   (r) & \texttt{TORRENT} (r)\\
\texttt{BMP}   (r) & \texttt{FLAC}  (r) & \texttt{MIFF}  (r) & \texttt{PFB}   (r) & \texttt{TTC}   (r)\\
\texttt{BPG}   (r) & \texttt{FLIF}  (r/w) & \texttt{MKA}   (r) & \texttt{PFM}   (r) & \texttt{TTF}   (r)\\
\texttt{BTF}   (r) & \texttt{FLV}   (r) & \texttt{MKS}   (r) & \texttt{PGF}   (r) & \texttt{TXT}   (r)\\
\texttt{CHM}   (r) & \texttt{FPF}   (r) & \texttt{MKV}   (r) & \texttt{PGM}   (r/w) & \texttt{VCF}   (r)\\
\texttt{COS}   (r) & \texttt{FPX}   (r) & \texttt{MNG}   (r/w) & \texttt{PLIST} (r) & \texttt{VRD}   (r/w/c)\\
\texttt{CR2}   (r/w) & \texttt{GIF}   (r/w) & \texttt{MOBI}  (r) & \texttt{PICT}  (r) & \texttt{VSD}   (r)\\
\texttt{CR3}   (r/w) & \texttt{GPR}   (r/w) & \texttt{MODD}  (r) & \texttt{PMP}   (r) & \texttt{WAV}   (r)\\
\texttt{CRM}   (r/w) & \texttt{GZ}    (r) & \texttt{MOI}   (r) & \texttt{PNG}   (r/w) & \texttt{WDP}   (r/w)\\
\texttt{CRW}   (r/w) & \texttt{HDP}   (r/w) & \texttt{MOS}   (r/w) & \texttt{PPM}   (r/w) & \texttt{WEBP}  (r)\\
\texttt{CS1}   (r/w) & \texttt{HDR}   (r) & \texttt{MOV}   (r/w) & \texttt{PPT}   (r) & \texttt{WEBM}  (r)\\
\texttt{CSV}   (r) & \texttt{HEIC}  (r/w) & \texttt{MP3}   (r)\footnotemark & \texttt{PPTX}  (r) & \texttt{WMA}   (r)\\
\texttt{DCM}   (r) & \texttt{HEIF}  (r/w) & \texttt{MP4}   (r/w) & \texttt{PS}    (r/w) & \texttt{WMV}   (r)\\
\texttt{DCP}   (r/w) & \texttt{HTML}  (r) & \texttt{MPC}   (r) & \texttt{PSB}   (r/w) & \texttt{WTV}   (r)\\
\texttt{DCR}   (r) & \texttt{ICC}   (r/w/c) & \texttt{MPG}   (r) & \texttt{PSD}   (r/w) & \texttt{WV}    (r)\\
\texttt{DFONT} (r) & \texttt{ICS}   (r) & \texttt{MPO}   (r/w) & \texttt{PSP}   (r) & \texttt{X3F}   (r/w)\\
\texttt{DIVX}  (r) & \texttt{IDML}  (r) & \texttt{MQV}   (r/w) & \texttt{QTIF}  (r/w) & \texttt{XCF}   (r)\\
\texttt{DJVU}  (r) & \texttt{IIQ}   (r/w) & \texttt{MRW}   (r/w) & \texttt{R3D}   (r) & \texttt{XLS}   (r)\\
\texttt{DLL}   (r) & \texttt{IND}   (r/w) & \texttt{MXF}   (r) & \texttt{RA}    (r) & \texttt{XLSX}  (r)\\
\texttt{DNG}   (r/w) & \texttt{INSP}  (r/w) & \texttt{NEF}   (r/w) & \texttt{RAF}   (r/w) & \texttt{XMP}   (r/w/c)\\
\texttt{DOC}   (r) & \texttt{INSV}  (r) & \texttt{NRW}   (r/w) & \texttt{RAM}   (r) & \texttt{ZIP}   (r)\\
\texttt{DOCX}  (r) & \texttt{INX}   (r) & \texttt{NUMBERS} (r) & \texttt{RAR}   (r) & \\
\texttt{DPX}   (r) & \texttt{ISO}   (r) & \texttt{O}     (r) & \texttt{RAW}   (r/w) & \\
\hline
\end{tabular}
\caption{File types supported by exiftool (v11.88) (r = read, w = write, c = create).}

\end{table}\footnotetext[4]{\label{org6806a28}You may notice that you can't write MP3 metadata. \texttt{ffmpeg} should be used instead.}

\section{ranger}
\label{sec:orga9d3806}
\href{https://github.com/ranger/ranger/}{ranger} is a free and open-source CLI files manager I'm using for years. It is very handy to select
images and preview them\footnote{As long as you use the right terminal emulator, \emph{e.g.} iTerm, kitty,
urxvt, xterm \emph{etc.} From my past experience, \href{https://github.com/elementary/terminal}{Elementary OS's pantheon-terminal} doesn't work.}. 

Append the following snippet in \texttt{\textasciitilde{}/.config/ranger/commands.py}.
It will add the custom command \texttt{rate\_image <0-5> <files>}.
\begin{minted}[]{python}
# ~/.config/ranger/commands.py
class rate_image(Command):
    """:rate_image <0-5> <files> 

    Command for rating image with exiftools.
    """

    def execute(self):
	import subprocess
	if self.arg(1):
	    rating_score = self.arg(1)
	else:
	    self.fm.notify("rate_img: a rating score is required!", bad=True)
	    return
	if self.arg(2):
	    files = self.arg(2)
	else:
	    cwd = self.fm.thisdir
	    cf = self.fm.thisfile
	    if not cwd or not cf:
		self.fm.notify("Error: no file selected for deletion!", bad=True)
		return
	    if len(cwd.marked_items) > 1:
		files = " ".join([f.shell_escaped_basename for f in cwd.marked_items])
		self.fm.mark_files(all=True, val=False)
	    else:
		files = cf.shell_escaped_basename
	command = "exiftool -rating=" + rating_score + \
		  " -overwrite_original_in_place " + files
	self.fm.notify("Run command: " + command)
	result = self.fm.execute_command(command, stdout=subprocess.PIPE)
	stdout, stderr = result.communicate()
	if result.returncode == 0:
	    # This is a generic function to print text in ranger.  
	    self.fm.notify("Succeed to rate image " + files + \
			   " with score " + rating_score + ".")
\end{minted}

It remains to define some key bindings to be granted the full power of \texttt{ranger}.
Append the following snippet to \texttt{\textasciitilde{}/.config/ranger/rc.conf}.

\begin{minted}[]{cfg}
# ~/.config/ranger/rc.conf
map r1 rate_image 1
map r2 rate_image 2
map r3 rate_image 3
map r4 rate_image 4
map r5 rate_image 5
\end{minted}

The resulting workflow is as follows:
\begin{itemize}
\item Select images with \texttt{SPC} (single selection) or \texttt{v} (reverse selection).
\item Press \texttt{r} and the rating \texttt{1} to \texttt{5}.
\end{itemize}
\url{ranger\_rating.gif}

\section{sxiv}
\label{sec:org32f7821}
\href{https://github.com/muennich/sxiv}{sxiv} is a free, open-source, lightweight and scriptable image viewer. 
Add the following entries in \texttt{\textasciitilde{}/.config/sxiv/exec/key-handler}\footnote{If this file doesn't exist, you can copy the sample one: \texttt{mkdir -p \textasciitilde{}/.config/sxiv/exec/ \&\& cp /usr/share/doc/sxiv/examples/key-handler .}}
\begin{minted}[]{shell}
# ~/.config/sxiv/exec/key-handler
case "$1" in
# ...
"C-1")      tr '\n' '\0' | xargs -0 exiftool -rating=1 -overwrite_original_in_place "{}" ;;
"C-2")      tr '\n' '\0' | xargs -0 exiftool -rating=2 -overwrite_original_in_place "{}" ;;
"C-3")      tr '\n' '\0' | xargs -0 exiftool -rating=3 -overwrite_original_in_place "{}" ;;
"C-4")      tr '\n' '\0' | xargs -0 exiftool -rating=4 -overwrite_original_in_place "{}" ;;
"C-5")      tr '\n' '\0' | xargs -0 exiftool -rating=5 -overwrite_original_in_place "{}" ;;
esac
\end{minted}

The resulting workflow is as follows:
\begin{itemize}
\item Open images with \texttt{sxiv}.
\item Press \texttt{C-x C-<1..5>} to rate the current image.
\item Or mark images with \texttt{m}\footnote{See \href{https://github.com/muennich/sxiv/blob/master/config.def.h}{sxiv/config.def.h} for more keybindings to mark images.}, toggle thumbnails mode with \texttt{RET} and press \texttt{C-x
  C-<1..5>} to rate selected images.
\end{itemize}
\section{Bonus}
\label{sec:org636e82c}
Some interesting tips are presented here.
The main dependencies are \texttt{zsh} and \texttt{GNU parallel}, adapt it
to fit your need.
\subsection{Rate or view images of specific rating}
\label{sec:org4d6f5b5}
The long command below search all JPG files (potentially symlink) of rating 3, 4
or 5 and view them with \texttt{sxiv}.
\begin{minted}[]{shell}
# 1. Find recursively all JPG under the current directory
# 2. Use GNU parallel, for each JPG 
#    (a) noglob (zsh specific): disable zsh globing to prevent "No such file" error
#    (b) Build the string "$Rating $Filepath"
#    (c) Remove the last line of exiftool output which count the amounts of the files.
#    (d) Keep entry with $Rating in {3, 4, 5}. Remove the first space.
# 3. Use GNU parallel: view at most 250 found images with sxiv at a time.
find -L . -type f -regex ".*\.jpg" | 
parallel -L 5000 \
  'noglob exiftool -p '\''$Rating $Filepath'\'' -f -q -q {} | \
   head --lines=-1 | \
   awk '\''$1 ~ /[345]/ {$1=""; print substr($0, 2)}'\'' ' |
parallel -L 250 -q sxiv "{}" 
\end{minted}
Some remarks:
\begin{itemize}
\item To see unrated images, use \texttt{-} as rating, then you can rate them with \texttt{sxiv} as above.
\item This one-shot command is fast enough for hundreds of images. Above this amount, you may
want to take some time to dump the result in a file as follows.
\end{itemize}
\begin{minted}[]{shell}
find -L -type f -regex ".*\.jpg$" | 
parallel -L 5000 \
 'noglob exiftool -p '\''$Rating $Filename'\'' -f -q -q {} | \
  head --lines=-1 | \
  awk '\''$1 ~ /[^0-]/ {print $0}'\'' ' >> ratingdb.txt
\end{minted}
And view images with specific rating with this command:
\begin{minted}[]{shell}
awk '$1 ~ /345/ {$1=""; print substr($0, 2)}' ratingdb.txt |
parallel -L 250 -q sxiv "{}" 
\end{minted}
\begin{itemize}
\item The best would be writing a python script to maintain an SQLite database, and
adapt the the script of ranger and sxiv above to update the database each time a
file rating changed.
\item \texttt{shuf} may be used after \texttt{awk} to shuffle images before viewing them.
\item Actually, no one will type again and again those lengthy commands. I either
use \texttt{C-R} in \texttt{zsh} with \href{https://github.com/junegunn/fzf}{fzf} for casual ones or add them as entries in \href{https://github.com/knqyf263/pet}{pet}, a
manager of parametrizable snippet.
\end{itemize}
\subsection{Migrate XMP rating into image metadata}
\label{sec:orgcff369a}
If you migrate from \texttt{darktable} or have XMP files with rating, you can try the following commands. 


\begin{minted}[]{shell}
zmodload zsh/mapfile
# 1. Find recursively all XMP under the current directory
# 2. Use GNU parallel, for each XMP 
#   (a) Extract the rating with grep and awk; put it in $RATE.
#   (b) Append the image path to .xmp-$RATE-rating.db
find . -type f -regex ".*\.xmp" -print0 | 
parallel -0  'TMP=$(grep -o "Rating=\"[012345]\"" {} | \
  awk '\''{print gensub(/Rating="([0-5])"/, "\1", "g", $1)}'\''); \
  echo {} >> xmp-$TMP-rating.db' 
# 3. For each $RATE: 
for i in {1..5}; do
    FNAME="xmp-$i-rating.db"
    # For each file in .xmp-$RATE-rating.db
    for f in "${(f)mapfile[$FNAME]}"; do   
	# Use exiftool to rate it with the corresponding $RATE
	exiftool -rating=$i -overwrite_original_in_place "${f%.*}"
    done 
done
\end{minted}
\end{document}
